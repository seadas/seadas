\documentclass[11pt, oneside]{article}   	% use "amsart" instead of "article" for AMSLaTeX format
\usepackage{color}
\input{rgb}
\usepackage{geometry}                		% See geometry.pdf to learn the layout options. There are lots.
\geometry{letterpaper}                   		% ... or a4paper or a5paper or ...
%\geometry{landscape}                		% Activate for for rotated page geometry
%\usepackage[parfill]{parskip}    		% Activate to begin paragraphs with an empty line rather than an indent
\usepackage{graphicx}				% Use pdf, png, jpg, or epsß with pdflatex; use eps in DVI mode
% \usepackage{makeindex}                   						% TeX will automatically convert eps --> pdf in pdflatex
\usepackage{amssymb}
\usepackage{tabularx}
\usepackage{fancyvrb}
\DefineVerbatimEnvironment{code}{Verbatim}{fontsize=\small}
\DefineVerbatimEnvironment{example}{Verbatim}{fontsize=\small}
\setlength{\extrarowheight}{3pt}
%\usepackage{hyperref}

\title{OCSSW Web Services Developer Manual}
\author{Aynur Abdurazik}
\date{\today}							% Activate to display a given date or no date


\begin{document}

\maketitle


\section*{\color{DarkGreen}Daily Progress}
\subsection*{\color{DeepPink3}To Do}
\begin{itemize}
\item Progress Monitor
\begin{itemize}
\item Save process input stream, error stream, and output stream on the server side database
\item Access process information from the client using pull mechanism
\item test running a few processors
\end{itemize}
\item Test l2gen
\item Test multi-level processor
\end{itemize}
\subsection*{\color{DarkTurquoise}Learning Objectives}
\begin{itemize}
\item Progress Monitor 
\item regular expression for single data with percentage information
\end{itemize}


\section{Packaging and Deployment}
\subsection{Packaging}

The \emph{\textbf{ocsww client-server module}}  is independent of the rest of SeaDAS. The jar file that will be deployed on the ocssw server is packaged using the following command in the \textdollar SEADAS\_HOME/seadas/seadas-ocsswrest directory:

\begin{code}
mvn install assembly:assembly
\end{code}

\section{\color{DarkOrange}OCSSW Server Side Representation}

\subsection{\color{DarkBlue} Execution}

The jar file is packaged on a developer machine and deployed on the server.

Run the following command from the command line to execute:
\begin{code}
java -Xmx2048m -jar seadas-ocsswserver-jar-with-dependencies.jar
\end{code}


\section{\color{DarkGreen}Virtual Box Configuration}

\subsection{Basic Configuration}
\begin{enumerate}
\item Need to install ``guest editions" to be able to resize the vm window.
\end{enumerate}

\subsection{File Sharing}
\begin{itemize}

\item Manually sharing a folder between host and guest machines: \\
In VirtualBox {\color{blue}{\bf Devices  $\rightarrow $  Shared Folder Settings...  $\rightarrow $ Shared Folders  $\rightarrow $  Machine Folders}}, select the folder from the host to be shared with the guest.
   \begin{enumerate}
      \item sudo rm /sbin/mount.vboxsf
      \item sudo ln -s /opt/VBoxGuestAdditions-4.3.20/lib/VBoxGuestAdditions/mount.vboxsf /sbin/mount.vboxsf
      \item mkdir ocsswws
      \item sudo mount -t vboxsf seadas-ocsswws /home/aabduraz/ocsswws
      \item To be able to write in the shared directory, it needs to be mounted in this way:\\
      sudo mount -t vboxsf -o uid=1000,gid=1000 seadas-ocsswrest /home/aabduraz/ocsswrest

   \end{enumerate}

\item Commands to manually mount a directory:
\begin{code}
sudo mount -t vboxsf seadas-ocsswrest /home/aabduraz/ocsswrest
\end{code}
where $seadas-ocsswrest$ is the name of folder, which has the development source code for web services, shared from the host machine, and $/home/aabduraz/ocsswrest$ is an empty folder in the virtual machine. The  $seadas-ocsswrest$ is shared to deploy the jar file from its $target$ directory after each build. 

\item need to install git
(error message:Error - Could not execute system command "git --version $>$ /dev/null" )

\end{itemize}

\subsection{Network Configuration}
\begin{enumerate}
\item The server must use $0.0.0.0$ as its IP address.
\item The client should still use $localhost$
\item The virtual machine uses ``NAT" port-forwarding, which is set through {\color{blue}{\bf Devices $\rightarrow$ Network $\rightarrow$ Network Settings ... }}.
\item  Between SeaDAS and OCSSWWS, we chose to use port number 6400 and 6401. The server side presents services using address ``$0.0.0.0:6401$", and a SeaDAS client will access the services using $``http://localhost:6400"$.
\end{enumerate}

\pagebreak

\section{\color{DarkOrange}Security Concepts and Implementation}
\subsection{\color{DarkBlue} Security Concepts}
\subsubsection{\color{DeepPink4}Java Keystore}


Java keystore is a repository of security certificates.
JDK provides a tool named $\{keytool\}$ to manipulates keystores. Java $keytool$ stores the keys and certificates in a keystore, protected by a keystore password.

\begin{quote}
TrustManager: Determines whether the remote authentication credentials \(and thus the connection\) should be trusted.

KeyManager: Determines which authentication credentials to send to the remote host.
\end{quote}

\subsubsection{\color{DeepPink4}Security Key Generation}
\begin{enumerate}
\item  Create a keystore for server
\begin{code}
keytool -genkey -alias server -keyalg RSA -keystore server.jks
\end{code}
My password for server keystore is "ocsswws". The generated file is $"server.jks"$.
\item  Create a keystore for client
\begin{code}
keytool -genkey -alias client -keyalg RSA -keystore client.jks
\end{code}
My password for server keystore is "ocsswwsclient". The generated file is $"client.jks"$.

\item
View the content of keystore files:
\begin{code}
keytool -list -v -keystore server.jks -storepass ocsswws
keytool -list -v -keystore client.jks -storepass ocsswwsclient
\end{code}
\item Get server's self signed public key certificate and store it in client's keystore.
\begin{code}
keytool -export -file server.cert -keystore server.jks -storepass ocsswws -alias server
\end{code}
\item Get client's self signed public key certificate and store it in server's keystore.
\begin{code}
keytool -export -file client.cert -keystore client.jks -storepass ocsswwsclient -alias client
\end{code}

{\color{red}Note: First we needed to export both server and client public key certificates into files.}
\item Use following commands to view certificate contents.
\begin{code}
keytool -printcert -v -file server.cert
keytool -printcert -v -file client.cert
\end{code}
\item As the last step, import server.cert into client keystore and client.cert into server keystore.
\begin{itemize}
\item store client's self signed public key certificate(client.cert) in server.jks against the alias "client".
\begin{code}
keytool -import -file client.cert -keystore server.jks -storepass ocsswws -alias client
\end{code}
\item store server.cert within client.jks against the alias "server".
\begin{code}
keytool -import -file server.cert -keystore client.jks -storepass ocsswwsclient -alias server
\end{code}
\end{itemize}

\item View the content of both keystore again using following commands.
\begin{code}
keytool -list -v -keystore server.jks -storepass ocsswws
keytool -list -v -keystore client.jks -storepass ocsswwsclient
\end{code}

\end{enumerate}

\subsubsection{\color{DeepPink4}Setting up SSL Configuration on OCSSW (Jersey) Client}
The SSL configuration is setup in the ClientBuilder class. The client builder contains methods for definition of KeyStore, TrustStore or entire SslContext.
\begin{itemize}
\item $KeyStore$ - Represents a storage facility for cryptographic keys and certificated; Manages different types of entries. The keystore in javax.net.ssl.keyStore contains  private keys and certificates.
\item $TrustStore$ -  The $javax.net.ssl.trustStore$ contain CA certificates that a server trusts when a remote party presents its certificate.
\item $SslContext$ -
\end{itemize}
\begin{code}
SslConfigurator sslConfig = SslConfigurator.newInstance()
    .trustStoreFile("truststore.jks")
    .trustStorePassword("asdfgh")
    .trustStoreType("JKS")
    .trustManagerFactoryAlgorithm("PKIX")

    .keyStoreFile("keystore.jks")
    .keyPassword("asdfgh")
    .keyStoreType("JKS")
    .keyManagerFactoryAlgorithm("SunX509")
    .keyStoreProvider("SunJSSE")

    .securityProtocol("SSL");

 SSLContext sslContext = sslConfig.createSSLContext();
\end{code}

\pagebreak


\section*{\color{DarkGreen}Daily Progress}
\subsection*{\color{DeepPink3}To Do}
\begin{itemize}
\item Confugure SSLContext and SslConfigurator
\begin{itemize}
\item have an architectural design document
\end{itemize}
\item Repackage ocsswrest
\item deploy on virtual box
\end{itemize}
\subsection*{\color{DarkTurquoise}Learning Objectives}
\begin{itemize}
\item SSLContext
\item SslConfigurator
\end{itemize}

\pagebreak

\section*{\color{DarkGreen}To Do List for OCSSW Web Services}

\begin{enumerate}
\item Figure out the secure communication configuration between server and client;
     \begin{itemize}
            \item document all concepts
            \item document directory structure and location of keystores
     \end{itemize}
\end{enumerate}

\pagebreak

\section*{\color{DarkRed}Resources}

\begin{enumerate}
\item This site provides information about problems with host identification: \\
\begin{verbatim}
 http://java.globinch.com/enterprise-java/
  security/fix-java-security-certificate-exception-no-matching-localhost-found/
 \end{verbatim}
\end{enumerate}

\end{document}