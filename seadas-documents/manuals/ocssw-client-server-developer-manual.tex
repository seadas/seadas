\documentclass[11pt, oneside]{article}   	% use "amsart" instead of "article" for AMSLaTeX format
\usepackage{color}
\input{rgb}
\usepackage{geometry}                		% See geometry.pdf to learn the layout options. There are lots.
\geometry{letterpaper}                   		% ... or a4paper or a5paper or ...
%\geometry{landscape}                		% Activate for for rotated page geometry
%\usepackage[parfill]{parskip}    		% Activate to begin paragraphs with an empty line rather than an indent
\usepackage{graphicx}				% Use pdf, png, jpg, or epsß with pdflatex; use eps in DVI mode
% \usepackage{makeindex}                   						% TeX will automatically convert eps --> pdf in pdflatex
\usepackage{amssymb}
\usepackage{tabularx}
\usepackage{fancyvrb}
\DefineVerbatimEnvironment{code}{Verbatim}{fontsize=\small}
\DefineVerbatimEnvironment{example}{Verbatim}{fontsize=\small}
\setlength{\extrarowheight}{3pt}
%\usepackage{hyperref}

\title{OCSSW Web Services Developer Manual}
\author{Aynur Abdurazik}
\date{\today}							% Activate to display a given date or no date


\begin{document}

\maketitle

\section{OCSSW}

\subsection{OCSSW Installation Locations}

There are four possible installation locations for OCSSW in the following order:
\begin{enumerate}
\item The path given as the value of seadas.ocssw.root variable in the seadas.config file
\item The value of the OCSSWROOT system variable specified in the script files
\item The seadas.home directory specified in the seadas.config file
\item user.home/ocssw directory
\end{enumerate}

These locations and their order of selection is implemented as the following:
\begin{code}
String ocsswInstallDirPath = RuntimeContext.getConfig().getContextProperty("ocssw.root", System.getenv("OCSSWROOT"));

                  if (dirPath == null) {
                      dirPath = RuntimeContext.getConfig().getContextProperty("home", System.getProperty("user.home") + File.separator + "ocssw");
\end{code}

\section{Packaging and Deployment}
\subsection{Packaging}

The \emph{\textbf{ocsww client-server module}}  is independent of the rest of SeaDAS. The jar file that will be deployed on the ocssw server is packaged using the following command in the \textdollar SEADAS\_HOME/seadas/seadas-ocsswrest directory:

\begin{code}
mvn install assembly:assembly
\end{code}

\subsection{Services and Application Description}

\subsubsection{Adding New REST Services}
How to register this new service?

\section{\color{DarkOrange}OCSSW Server Side Representation}

\subsection{\color{DarkBlue} OCSSW programs}


\subsection{\color{DarkBlue} Execution}

The jar file is packaged on a developer machine and deployed on the server.

Run the following command from the command line to execute:
\begin{code}
java -Xmx2048m -jar seadas-ocsswserver-jar-with-dependencies.jar
\end{code}


\section{\color{DarkGreen}Virtual Box Configuration}

\subsection{Basic Configuration}
\begin{enumerate}
\item Need to install ``guest editions" to be able to resize the vm window and also to enable file sharing between the host and the guest.
\item Choose "Fixed" virtual box for performance. Eloborate more later.
\item Need to install Java to run the ocssw rest server.
\end{enumerate}

\subsubsection{How to install guest additions}
The guest additions are actually included (invisibly) in the binary package. You get at them by choosing Install Guest Additions from the Devices menu.
You then install them by mounting the Guest Additions iso file,
which will show up in the list of available images. The Guest Additions are not available separately and don't need to be.

\subsubsection{Java Installation on the Ubuntu Linux Virtual Box}

\begin{enumerate}
\item Check to see if your Ubuntu Linux operating system archtecture is 32-bit or 64-bit, open up a terminal and run the following command below:
     \begin{code}
             file /sbin/
      \end{code}
\item
\end{enumerate}
\subsection{File Sharing}

Usage of disk space. Does it take virtual box disk space? How to increase disk space when needed for "fixed" virtual boxes?

How disk space is used on virtual machine, i.e., how is it allocated?

error message when no disk space left:
"segmentation error"

error message for incorrect way of increasing disk space:

VBoxManage modifyhd "/Users/aabduraz/VirtualBox VMs/OCSSW_Ubuntu_Server/OCSSW_Ubuntu_Server.vdi" --resize 81920
0%...
Progress state: VBOX_E_NOT_SUPPORTED
VBoxManage: error: Resize hard disk operation for this format is not implemented yet!

The reason for the error was there are two disk space measures, virtual size and actual size. The resize amount has to be greater than the virtual size specified in the "Storage" configuration of the virtual machine.

\begin{itemize}

\item Manually sharing a folder between host and guest machines: \\
In VirtualBox {\color{blue}{\bf Devices  $\rightarrow $  Shared Folder Settings...  $\rightarrow $ Shared Folders  $\rightarrow $  Machine Folders}}, select the folder from the host to be shared with the guest.
   \begin{enumerate}
      \item sudo rm /sbin/mount.vboxsf
      \item sudo ln -s /opt/VBoxGuestAdditions-4.3.20/lib/VBoxGuestAdditions/mount.vboxsf /sbin/mount.vboxsf
      \item mkdir ocsswws
      \item sudo mount -t vboxsf seadas-ocsswws /home/aabduraz/ocsswws
      \item To be able to write in the shared directory, it needs to be mounted in this way:\\
      sudo mount -t vboxsf -o uid=1000,gid=1000 seadas-ocsswrest /home/aabduraz/ocsswrest

   \end{enumerate}

\item Commands to manually mount a directory:
\begin{code}
sudo mount -t vboxsf seadas-ocsswrest /home/aabduraz/ocsswrest
\end{code}
where $seadas-ocsswrest$ is the name of folder, which has the development source code for web services, shared from the host machine, and $/home/aabduraz/ocsswrest$ is an empty folder in the virtual machine. The  $seadas-ocsswrest$ is shared to deploy the jar file from its $target$ directory after each build. 

\item need to install git
(error message:Error - Could not execute system command "git --version $>$ /dev/null" )

\end{itemize}

\subsection{Network Configuration}
\begin{enumerate}
\item The server must use $0.0.0.0$ as its IP address.
\item The client should still use $localhost$
\item The virtual machine uses ``NAT" port-forwarding, which is set through {\color{blue}{\bf Devices $\rightarrow$ Network $\rightarrow$ Network Settings ... }}.
\item  Between SeaDAS and OCSSWWS, we chose to use port number 6400 and 6401. The server side presents services using address ``$0.0.0.0:6401$", and a SeaDAS client will access the services using $``http://localhost:6400"$.
\end{enumerate}

\pagebreak

\section{\color{DarkOrange}Security Concepts and Implementation}
\subsection{\color{DarkBlue} Security Concepts}
\subsubsection{\color{DeepPink4}Java Keystore}


Java keystore is a repository of security certificates.
JDK provides a tool named $\{keytool\}$ to manipulates keystores. Java $keytool$ stores the keys and certificates in a keystore, protected by a keystore password.

\begin{quote}
TrustManager: Determines whether the remote authentication credentials \(and thus the connection\) should be trusted.

KeyManager: Determines which authentication credentials to send to the remote host.
\end{quote}

\subsubsection{\color{DeepPink4}Security Key Generation}
\begin{enumerate}
\item  Create a keystore for server
\begin{code}
keytool -genkey -alias server -keyalg RSA -keystore server.jks
\end{code}
My password for server keystore is "ocsswws". The generated file is $"server.jks"$.
\item  Create a keystore for client
\begin{code}
keytool -genkey -alias client -keyalg RSA -keystore client.jks
\end{code}
My password for server keystore is "ocsswwsclient". The generated file is $"client.jks"$.

\item
View the content of keystore files:
\begin{code}
keytool -list -v -keystore server.jks -storepass ocsswws
keytool -list -v -keystore client.jks -storepass ocsswwsclient
\end{code}
\item Get server's self signed public key certificate and store it in client's keystore.
\begin{code}
keytool -export -file server.cert -keystore server.jks -storepass ocsswws -alias server
\end{code}
\item Get client's self signed public key certificate and store it in server's keystore.
\begin{code}
keytool -export -file client.cert -keystore client.jks -storepass ocsswwsclient -alias client
\end{code}

{\color{red}Note: First we needed to export both server and client public key certificates into files.}
\item Use following commands to view certificate contents.
\begin{code}
keytool -printcert -v -file server.cert
keytool -printcert -v -file client.cert
\end{code}
\item As the last step, import server.cert into client keystore and client.cert into server keystore.
\begin{itemize}
\item store client's self signed public key certificate(client.cert) in server.jks against the alias "client".
\begin{code}
keytool -import -file client.cert -keystore server.jks -storepass ocsswws -alias client
\end{code}
\item store server.cert within client.jks against the alias "server".
\begin{code}
keytool -import -file server.cert -keystore client.jks -storepass ocsswwsclient -alias server
\end{code}
\end{itemize}

\item View the content of both keystore again using following commands.
\begin{code}
keytool -list -v -keystore server.jks -storepass ocsswws
keytool -list -v -keystore client.jks -storepass ocsswwsclient
\end{code}

\end{enumerate}

\subsubsection{\color{DeepPink4}Setting up SSL Configuration on OCSSW (Jersey) Client}
The SSL configuration is setup in the ClientBuilder class. The client builder contains methods for definition of KeyStore, TrustStore or entire SslContext.
\begin{itemize}
\item $KeyStore$ - Represents a storage facility for cryptographic keys and certificated; Manages different types of entries. The keystore in javax.net.ssl.keyStore contains  private keys and certificates.
\item $TrustStore$ -  The $javax.net.ssl.trustStore$ contain CA certificates that a server trusts when a remote party presents its certificate.
\item $SslContext$ -
\end{itemize}
\begin{code}
SslConfigurator sslConfig = SslConfigurator.newInstance()
    .trustStoreFile("truststore.jks")
    .trustStorePassword("asdfgh")
    .trustStoreType("JKS")
    .trustManagerFactoryAlgorithm("PKIX")

    .keyStoreFile("keystore.jks")
    .keyPassword("asdfgh")
    .keyStoreType("JKS")
    .keyManagerFactoryAlgorithm("SunX509")
    .keyStoreProvider("SunJSSE")

    .securityProtocol("SSL");

 SSLContext sslContext = sslConfig.createSSLContext();
\end{code}


\section{Virtualbox Related FAQ}
A Virtual Box Ubuntu behaves very similar to Ubuntu on bare metal. Therefore the solutions from this question

How to fix "The system is running in low-graphics mode" error?
mostly hold true as well. However note, that you do not have a graphics card or proprietary drivers in a VM. This will all be handled by the guest additions.

We can enter a TTY for command line repair by press and hold the HOST-key, which in your case is the right Ctrl-key and simultanously press F1 trough F8.

In case your virtual hard drive runs short of memory try first with

sudo apt-get autoremove
sudo apt-get autoclean
to remove no longer needed packages.


\subsection{How to free up disk space}

\begin{itemize}
\item empty the directory /tmp
\item dpkg -l | grep ^rc | cut -d ' ' -f3 | xargs sudo apt-get purge -y
\item sudo apt-get autoremove
\item Synaptic
\item sudo apt-get clean
\ietm sudo apt-get autoclean
\item sudo apt-get remove --purge linux-image-X.X.XX-XX-generic    -- delete old kernels
to decide which kernel version to remove: dpkg --get-selections | grep linux-image
\end{itemize}

\end{document}
